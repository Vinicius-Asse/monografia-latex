Em 2018, a NVIDIA anunciou um novo módulo programável para suas GPUs capaz de aprimorar o fluxo de renderização atual denominado de mesh shaders. Com o intuito de explorar suas novas funcionalidades, estudamos seu uso um contexto de geração de geometria procedural. Para isto, desenvolvemos uma ferramenta com capacidade de compilar e executar mesh shaders. Com o auxilio dessa ferramenta, pudemos observar a performance e o uso dos recursos durante a execução do algoritmo de marching cubes. Por fim, comparamos os resultados obtidos com a implementação do mesmo algoritmo em compute shaders.

Palavras chave: mesh shaders, geometria procedural, marching cubes.